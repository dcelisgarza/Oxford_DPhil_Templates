% Author: Daniel Celis Garza <daniel.celisgarza@materials.ox.ac.uk>

% Preamble with all the basic packages a thesis would need. Modify as needed.

\documentclass[twoside,htwologo,twosup,12pt]{genthesis}
% By changing the document class this preamble can be used for other document types.
% genthesis.cls contains the definition of the document class.
% Geometry
\usepackage[top=1in, bottom=1in,
	outer=1in, inner=1.5in] % inner = 1.5 for binding
{geometry}

%----------------------- Fonts, symbols and colours ------------------------%
% If using pdfLaTeX comment fontspec and uncomment fontenc and inputenc. If using the superior XeLaTeX/XeTeX or LuaTeX do the opposite.
\usepackage{fontspec}					% XeLaTeX & LuaTeX fonts.
%\usepackage[T1]{fontenc}				% Font encoding for pdfLaTeX
%\usepackage[latin1]{inputenc}			% Input encoding (easy accents) for pdfLaTeX
\usepackage{amssymb, amsmath,
	bm	   , isomath,
	mathtools, esint}					% Maths fonts and symbols.
\usepackage{exscale}					% Removes the need to use {\displaystyle }.
%\newfontfamily\ubuntumono{Ubuntu Mono} % Ubuntu font for fancy shell commands (font needs to be installed).
\usepackage{xcolor}
%\definecolor{oxfordblue}{cmyk}{79,56,0,72}
\definecolor{oxfordblue}{RGB}{15,31,71}
\usepackage{lmodern}

%------------------------ Hyperlinks and references ------------------------%
\usepackage[colorlinks     = true, % Hyperlinks.
	pdfstartview   = FitV,
	linkcolor      = oxfordblue,
	citecolor      = oxfordblue,
	urlcolor       = oxfordblue,
	hyperfootnotes = true,
	hypertexnames  = true,
	plainpages     = false % Correctly links index entries whenever \thispagestyle{empty} is used.
]{hyperref}
\usepackage[comma  , square,       % Citing style.
	numbers, sort&compress
]{natbib}
\usepackage{cleveref} 			   % Automatic referencing better than \autoref{}.

%----------------------------- Macro utility -------------------------------%
\usepackage{xparse}	 % Expanded macro capability.

%--------------------------- Index and glossary ----------------------------%
\usepackage{makeidx} % Index.
\makeindex			 % Creates index.
\usepackage[toc, acronyms
]{glossaries} % Glossary.
\setglossarystyle{altlisthypergroup} % Sets a list of letters with hyperlinks before the glossaries and acronyms.
% Dual glossary entry.
% Define dual entry for glossary + acronym.
% https://en.wikibooks.org/wiki/LaTeX/Glossary#Dual_entries_with_reference_to_a_glossary_entry_from_an_acronym
\DeclareDocumentCommand{\newdualentry}{ O{} O{} m m m m } {
	\newglossaryentry{gls-#3}{name={#5},text={#5\glsadd{#3}},
	description={#6},#1
}
\makeglossaries
\newacronym[see={[Glossary:]{gls-#3}},#2]{#3}{#4}{#5\glsadd{gls-#3}}
}

%--------------------------------- Utility ---------------------------------%
\usepackage{setspace} % Text spacing commands.
\usepackage{paralist} % In-paragraph lists.
%\usepackage{pdfpages} % Include pdf pages.
%\usepackage{lscape}   % Use landscape pages.
%\allowdisplaybreaks   % Math environments continue onto the next page if they overflow.
\usepackage{siunitx}  % International units.
\usepackage{hologo}   % XeLaTeX and BibTex logos.
\usepackage{ragged2e} % Ragged text.

%--------------------------------- Floats ----------------------------------%
%\usepackage{float} 			% Extra options for floats.
%\usepackage[section]{placeins} % Force floats to stay in the sections they're called in.
\usepackage{booktabs} 			% Nicer tables.
%\usepackage{multirow} 			% Multirow and multicolumn tables. Avoid when possible.
\usepackage{subcaption} 		% Subfigures.
%\usepackage{epstopdf} 			% pdfLaTeX does not support eps images so they need to be converted to pdf.

%---------------------------- Scripts and code -----------------------------%
% If you only need basic script support use verbatim. If you want more features use listings. If you can install pygments use minted.
%\usepackage{verbatim} % Type commands without the hassle of the other two.
%\usepackage{listings} % Spartan display of code and pseudo code.
\usepackage{minted}   % Elegantly display code. Requires a Python 2.7 or higher installation of pygments to be installed. Requires "-shell-escape" flag to the LaTeX compilation command.
\usepackage[chapter]{algorithm}
\usepackage{algpseudocode} % Package for algorithm typesetting (pseudo-code).

%---------------------------- Image file paths -----------------------------%
\graphicspath{{../images/}}

%------------------------------- Input files -------------------------------%
\newcommand{\kwdmc}[2]{\gls{#1}\index{\ensuremath{#2}}} % Keyword math command.
\newcommand{\kwdm}[1]{\gls{#1}\index{\ensuremath{#1}}}  % Keyword math.
\newcommand{\kwd}[1]{\gls{#1}\index{#1}}				% Keyword.
\newcommand{\incarabcounter}{							% Preserves arabic numbering before using the romanpages environment.
	\setcounter{arabiccounter}{\value{page}}
	%
	\if@twocolumn
		\addtocounter{arabiccounter}{1}
	\else
		\if@openright
			\addtocounter{arabiccounter}{2}
		\else
			\addtocounter{arabiccounter}{1}
		\fi
	\fi
}	 % Macros.
\newglossaryentry{pi}
{
	name = {\ensuremath{\pi}},
	description = {Ratio of circumference of circle to its
		diameter},
	sort = pi
}
\newglossaryentry{e}
{
	name = {\ensuremath{e}},
	description = {Euler's number defined as \ensuremath{\lim\limits_{n\to\infty} \left(1 + \dfrac{1}{n}\right)^{n}}}
}
\newglossaryentry{tensor}
{
	name={tensor},
	description={Geometric object that describes linear relations between geometric vectors, scalars and other tensors. They are generalisations of scalars (no indices), vectors (one index) and matrices (two indices) to $ n $ indices},
	plural=tensors
} % Glossaries.

%----------------------------- Special options -----------------------------%
%\counterwithout{figure}{chapter} % Uncomment to remove chapter number from figures.
%\counterwithout{table}{chapter}  % Uncomment to remove chapter number from tables.
%\setlength{\topskip}{0in}        % Top paragraph spacing.
%\setlength{\parskip}{2ex}        % Bottom paragraph spacing.
%\linespread{1.5} 				  % Inter-line spacing (1 = single space, 1.5 = double space).
%\renewcommand{\footnoterule}{% Footnote fule that spans the whole textwidth.
%	\kern -5.4pt
%	\hrule width \textwidth height 0.4pt
%	\kern 5pt
%}
%\makeatletter \newcommand*{\codefont}{% \codefont provides font size for code (for use in listings and minted).
%	\@setfontsize
%	\codefont{10pt}{11pt}
%}\makeatother