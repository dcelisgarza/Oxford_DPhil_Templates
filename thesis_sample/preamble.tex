% Author: Daniel Celis Garza <daniel.celisgarza@materials.ox.ac.uk>
% Date of creation: 2017/03/09
% Last edit: 2017/03/09

% Preamble with all the basic packages a thesis would need. Modify as needed.

\documentclass[htwologo,twosup,12pt]{genthesis}
% By changing the document class this preamble can be used for other document types.
% genthesis.cls contains the definition of the document class.
% Geometry
\usepackage[top=1in, bottom=1in, 
			outer=1in, inner=1in] % inner = 1.5 for binding
			{geometry}

%---------------------------- Fonts and symbols ----------------------------%
% If using pdfLaTeX comment fontspec and uncomment fontenc and inputenc. If using the superior XeLaTeX/XeTeX or LuaTeX do the opposite.
\usepackage{fontspec}					% XeLaTeX & LuaTeX fonts.
%\usepackage[T1]{fontenc}				% Font encoding for pdfLaTeX
%\usepackage[latin1]{inputenc}			% Input encoding (easy accents) for pdfLaTeX
\usepackage{amssymb,amsmath,bm} 		% Maths fonts and symbols.
%\usepackage{exscale}					% Removes the need to use {\displaystyle }.
%\newfontfamily\ubuntumono{Ubuntu Mono} % Ubuntu font for fancy shell commands (font needs to be installed).

%------------------------ Hyperlinks and references ------------------------%
\usepackage[colorlinks     = true, % Hyperlinks.
			pdfstartview   = FitV,
			linkcolor      = blue,
			citecolor      = blue,
			urlcolor       = blue,
			hyperfootnotes = true,
			hypertexnames  = true,
			plainpages     = false % Correctly links index entries whenever \thispagestyle{empty} is used.
			]{hyperref}
\usepackage[comma, square,         % Citing style.
			numbers, sort&compress
			]{natbib}
\usepackage{cleveref} 			   % Automatic referencing better than \autoref{}.

%--------------------------- Index and glossary ----------------------------%
\usepackage{makeidx} 	 % Index.
\makeindex				 % Creates index.
\usepackage[toc]{glossaries}	 % Glossary.
\newglossaryentry{pi}
{
	name = {\ensuremath{\pi}},
	description = {Ratio of circumference of circle to its
		diameter},
	sort = pi
}
\newglossaryentry{e}
{
	name = {\ensuremath{e}},
	description = {Euler's number defined as \ensuremath{\lim\limits_{n\to\infty} \left(1 + \dfrac{1}{n}\right)^{n}}}
}
\newglossaryentry{tensor}
{
	name={tensor},
	description={Geometric object that describes linear relations between geometric vectors, scalars and other tensors. They are generalisations of scalars (no indices), vectors (one index) and matrices (two indices) to $ n $ indices},
	plural=tensors
}	 % Inputs the glossary file.
\makeglossaries

%--------------------------------- Utility ---------------------------------%
\usepackage{setspace}
%\usepackage{paralist} % In-paragraph lists.
%\usepackage{pdfpages} % Include pdf pages.
%\usepackage{lscape}   % Use landscape pages.
%\allowdisplaybreaks   % Math environments continue onto the next page if they overflow.

%--------------------------------- Floats ----------------------------------%
%\usepackage{float} 			% Extra options for floats.
%\usepackage[section]{placeins} % Force floats to stay in the sections they're called in.
\usepackage{booktabs} 			% Nicer tables.
%\usepackage{multirow} 			% Multirow and multicolumn tables. Avoid when possible.
\usepackage{subcaption} 		% Subfigures.
%\usepackage{epstopdf} 			% pdfLaTeX does not support eps images so they need to be converted to pdf.

%---------------------------- Scripts and code -----------------------------%
% If you only need basic script support use verbatim. If you want more features use listings. If you can install pygments use minted.
%\usepackage{verbatim} % Type commands without the hassle of the other two.
%\usepackage{listings} % Spartan display of code and pseudo code.
%\usepackage{minted}   % Elegantly display code and pseudo code. Requires a Python 2.7 or higher installation of pygments to be installed. Requires "-shell-escape" flag to the compilation command.

%------------------------------- File paths --------------------------------%
\graphicspath{{./images/}} % Every N "../" represents going up N-1 folders before going to the folder following the /

%----------------------------- Special options -----------------------------%
%\counterwithout{figure}{chapter} % Uncomment to remove chapter number from figures.
%\counterwithout{table}{chapter}  % Uncomment to remove chapter number from tables.
%\setlength{\topskip}{0in}        % Top paragraph spacing.
%\setlength{\parskip}{2ex}        % Bottom paragraph spacing.
%\linespread{1.5} 				  % Inter-line spacing (1 = single space, 1.5 = double space).
%\renewcommand{\footnoterule}{% Footnote fule that spans the whole textwidth.
%	\kern -5.4pt
%	\hrule width \textwidth height 0.4pt
%	\kern 5pt
%}
%\makeatletter \newcommand*{\codefont}{% \codefont provides font size for code (for use in listings and minted).
%	\@setfontsize
%	\codefont{10pt}{11pt}
%}\makeatother

%--------------------------- User-defined macros ---------------------------%
%--------------------------------- Colours ---------------------------------%
\definecolor{oxfordblue}{RGB}{15,31,71}
\definecolor{silver}{RGB}{240,240,240}

%---------------------------------- Fonts ----------------------------------%
% Section fonts.
\sectionfont{\fontsize{60}{72}\color{oxfordblue}\selectfont}
% Caption fonts.
\DeclareCaptionFont{large}{\fontsize{37}{44.4}\selectfont}
\DeclareCaptionFont{oxfordblue}{\color{oxfordblue}}
\captionsetup{
	font 	  = large,
	labelfont = {color=oxfordblue}
}

%------------------------ Header and Footer Images -------------------------%
% Header image.
\newcommand{\himage}[2]{
	\begin{tikzpicture}[overlay, remember picture]
		\node[anchor=north] at (current page.north) {\includegraphics[width=#1]{#2}};
	\end{tikzpicture}
}
% Footer image.
\newcommand{\fimage}[2]{
	\begin{tikzpicture}[overlay, remember picture]
		\node[anchor=south] at (current page.south) {\includegraphics[width=#1]{#2}};
	\end{tikzpicture}	
}

%--------------------------------- Images ----------------------------------%
% They are not environments because environments break with more than 2 arguments
% and they aren't flexible enough to provide the scfloat structure that I want.
% Centre float.
\newcommand{\cfloat}[4]{%
	\vspace{#1}
	\begin{center}% Centre minipage.
		\begin{minipage}{#2}
			\centering% Centre minipage content.
			#3
		\end{minipage}
	\end{center}
	\vspace{#4}
}
% Side caption float.
\newcommand{\scfloat}[7]{%
	\vspace{#1}
	\begin{center}% Center minipage.
		\begin{minipage}{#2}% Left side.
			\centering% Centre minipage content.
			#3
		\end{minipage}%
		#4
		\begin{minipage}{#5}% Right side.
			\centering% Centre minipage content.
			#6
		\end{minipage}%
	\end{center}
	\vspace{#7}
}